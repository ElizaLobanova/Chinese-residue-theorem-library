\documentclass[10pt]{article}
\usepackage{femj_ru}

%% Перед отправкой в журнал:
% 1. Перевести файл в кодировку windows-1251
% 2. в файле femj_ru.sty поменять \RequirePackage[utf8]{inputenc} на \RequirePackage[cp1251]{inputenc}
% 3. в файле femj_ru.sty удалить строку 10 \RequirePackage{soulutf8}

%Глубокоуважаемые авторы!

% прежде чем подключать дополнительные пакеты TeXa и создавать новые окружения, пожалуйста, ознакомьтесь с уже предусмотренными в стилевом файле журнала

% Используемые пакеты  перечислены в файле femj_ru.sty строки 7-30
% Обозначения окружений журнала и дополнительные команды приведены в стилевом файле femj_ru.sty строки 38-80

\begin{document}

\Pages(0--0)

\def\Im{\mathop{\mathrm{Im}}\nolimits}

\summary Trukhin~V.\,O.$^{1, 2}$\!, Strongin~V.\,S.$^{1, 2}$\!, Lobanova~E.\,A.$^{1, 2}$\, Anisich~A.\,I.$^{2}$\author

Application of the Chinese residue theorem for dealing with large numbers in supercomputing\title

In this paper we consider Something new and cool!
\keywords{Metropolis algorithm, statistical thermodynamics.}
\org{$^1$ Institute for Applied Mathematics, Far Eastern Branch, Russian Academy of Sciences\\ 
$^2$ Department of theoretical physics, Far eastern federal university, Russia}
\references{ %транслитерированный список литературы

\begin{thebibliography}{4}
\setlength{\parsep}{0pt}\setlength{\itemsep}{3pt}

\bibitem{shevchenko2017}
\by Y. Shevchenko, A. Makarov, K. Nefedev
\jour Physics Letters A
\paper Effect of long-and short-range interactions on the thermodynamics of dipolar spin ice
\vol 381
\issue 5
\pages 428-434
\yr 2017

\end{thebibliography}
}

\UDC{511.21+517.965+517.547.582}
\AMS{11B37 + 33E05}

\SupportedBy{Исследование выполнено за счет гранта Российского научного фонда № 23-22-00328, \linebreak https://rscf.ru/project/23-22-00328/\linebreak 
Представленные в работе результаты были получены на суперкомпьютерном вычислительном кластере Института прикладной математики ДВО РАН.
}

\submitted{16 сентября 2024 г.}

\title{Применение китайской теоремы об остатках для работы с большими числами в суперкомпьютерных вычислениях}

\author[1,2]{В.\, О.~Трухин}{Департамент теоретической физики и интеллектуальных технологий, Институт наукоемких технологий и передовых материалов, Дальневосточный федеральный университет. 690922, Россия, г. Владивосток, о. Русский, п. Аякс, 10}{}
\author[1,2]{В.\, С.~Стронгин}{Институт прикладной математики, Дальневосточное отделение Российской академии наук. 690041, Россия, г. Владивосток, Ул. Радио д. 7}{}
\author[1,2]{Э.\, А.~Лобанова}{}{}
\author[2]{А.\, И.~Анисич}{}{anisich.ai@dvfu.ru}




\makeface

%\markright {Разбавленная модель кубического спинового льда ...} %Добавляем, если длинное название статьи
\markleft{В.\,С. Стронгин, П.\,А. Овчинников, Э.\,А.~Лобанова, ...} %добавляем, если много авторов


\abstract Задача ис

\keywords{алгоритмы, статистическая термодинамика.}

\DOI{to be presented}

\section*{Введение}

Во ряде задачах теоретической физики, аналитической биологии (?),.............. данные представляются в виде чисел превышающих стандартные типы языков программирования ($\approx 2^{64}$). Традиционно эту проблема решается представлением чисел в виде строки произвольной длины (GNU Multi-Precision Library, Boost и др.). Однако такой подход замедляет расчёты, и не работает с графическими процессорами из-за их архитектуры.

В статье представлен алгоритм разложения больших чисел на остатки от деления на набор простых чисел, математические операции над полученными массивами и способ обратной сборки.

\section{Математическая модель} 

Тут текст и ссылка \cite{shevchenko2017}

\section {Теплоемкость кубического спинового льда}

Ещё один раздел

\section{Заключение}

Заключение по публикации

%\vspace{15mm}

\begin{thebibliography}{20}
\setlength{\parsep}{0pt}\setlength{\itemsep}{3pt}

\bibitem{shevchenko2017}
\by Y. Shevchenko, A. Makarov, K. Nefedev
\jour Physics Letters A
\paper Effect of long-and short-range interactions on the thermodynamics of dipolar spin ice
\vol 381
\issue 5
\pages 428-434
\yr 2017

\end{thebibliography}



\EndArticle
\end{document} 