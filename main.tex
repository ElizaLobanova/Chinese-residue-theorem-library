\documentclass{crm-article}
\usepackage{algorithm}
\usepackage{algorithmic}

\begin{document}

\journalVol{10}
\journalNo{1} %выпуска
\setcounter{page}{1}

% раздел журнала
\journalSection{Математические основы и численные методы моделирования}
\journalSectionEn{Mathematical modeling and numerical simulation}


% дата получения
\journalReceived{01.06.2016.}
%\journalReviewed{01.06.2016.}
%принято к публикации
\journalAccepted{01.06.2016.}


\UDC{519.8}
\title{Применение китайской теоремы об остатках для работы с длинной арифметикой}
\titleeng{Application of the Chinese Remainder Theorem to Multi-Precision Computations}
\thanks{Исследование выполнено за счет гранта Российского научного фонда № 24-71-10069, \linebreak https://rscf.ru/project/24-71-10069. \linebreak}
\thankseng{The research was supported by the Russian Science Foundation grant No. 24-71-10069, \linebreak https://rscf.ru/en/project/24-71-10069. \linebreak}

\author[1,2]{\firstname{В.\,О.}~\surname{Трухин}}
\authorfull{Вячеслав Олегович Трухин}
\authoreng{\firstname{V.\,O.}~\surname{Trukhin}}
\authorfulleng{Viacheslav O. Trukhin}
\email{slavatruhin@gmail.com}
\affiliation[1]{Департамент теоретической физики и интеллектуальных технологий, Институт наукоемких технологий и передовых материалов, Дальневосточный федеральный университет,\protect\\ 690922, Россия, г. Владивосток, о. Русский, п. Аякс, 10}
\affiliation[2]{Институт прикладной математики, Дальневосточное отделение Российской академии наук,\protect\\ 690041, Россия, г. Владивосток, Ул. Радио д. 7}
\affiliationeng[1]{Department of theoretical physics and intellectual technologys, Institute of scientific and advanced materials, Far eastern federal university,\protect\\ Address}
\affiliationeng[2]{Institute for Applied Mathematics, Far Eastern Branch, Russian Academy of Sciences,\protect\\ Address}

\author[1,2]{\firstname{В.\,С.}~\surname{Стронгин}}
\authorfull{Владислав Сергеевич Стронгин}
\authoreng{\firstname{V.\,O.}~\surname{Strongin}}
\authorfulleng{Vladislav S. Strongin}

\author[1,2]{\firstname{Э.\,А.}~\surname{Лобанова}}
\authorfull{Элиза Александровна Лобанова}
\authoreng{\firstname{E.\,A.}~\surname{Lobanova}}
\authorfulleng{Eliza A. Lobanova}

\author[1,2]{\firstname{А.\,Г.}~\surname{Макаров}}
\authorfull{Александр Геннадьевич Макаров}
\authoreng{\firstname{A.\,G.}~\surname{Makarov}}
\authorfulleng{Aleksandr G. Makarov}

\author[1,2]{\firstname{К.\,В.}~\surname{Нефедев}}
\authorfull{Нефедев Константин Валентинович}
\authoreng{\firstname{K.\,V.}~\surname{Nefedev}}
\authorfulleng{Konstantin V. Nefedev}

\begin{abstract}
В работе рассматривается применение Китайской теоремы об остатках (КТО) для повышения эффективности вычислений с большими целыми числами в задачах высокопроизводительных вычислений. Актуальность исследования обусловлена тем, что в ряде задач теоретической физики, численного моделирования и переборных алгоритмов используются числа разрядности, существенно превышающей возможности стандартных машинных типов данных, что приводит к необходимости применения длинной арифметики и сопровождается значительными вычислительными затратами. Традиционные методы представления длинных чисел в виде последовательностей слов плохо масштабируются и недостаточно эффективно используют архитектуру графических процессоров (GPU).

Предложенный подход основан на представлении числа в системе вычетов по набору попарно взаимно простых модулей. Такое разложение позволяет заменить операции над многобайтными числами на независимые модульные операции над массивами остатков, что обеспечивает возможность их естественного распараллеливания. Арифметические операции сложения, вычитания и умножения выполняются покомпонентно в модульной системе, а восстановление исходного числа осуществляется с использованием алгоритмов, основанных на расширенном алгоритме Евклида. Показано, что при последовательных вычислениях предложенный метод не имеет существенного преимущества по сравнению с классической длинной арифметикой, однако при реализации на GPU достигается значительный рост производительности за счёт массового параллелизма и независимости операций по модулям. Дополнительным достоинством является снижение риска переполнения и более эффективная организация доступа к памяти благодаря использованию чисел фиксированной разрядности.

Таким образом, использование КТО представляет собой эффективный инструмент для реализации многоточечных вычислений с большими числами в суперкомпьютерных системах и может применяться в задачах численного моделирования, криптографии и алгоритмах полного перебора, требующих высокой производительности и масштабируемости.
\end{abstract}


\keyword{большие числа}
\keyword{суперкомпьютеры}
\keyword{модульная арифметика}
\keyword{параллельные вычисления}

\begin{abstracteng}
This work investigates the application of the Chinese Remainder Theorem (CRT) to improve the efficiency of computations involving large integers in high-performance computing tasks. The relevance of this study arises from the fact that a number of problems in theoretical physics, numerical simulation, and exhaustive search algorithms require integers whose bit length significantly exceeds the capabilities of standard machine data types. This necessitates the use of arbitrary-precision arithmetic and leads to substantial computational overhead. Traditional representations of large integers as sequences of machine words scale poorly and do not efficiently utilize the architecture of graphics processing units (GPUs).

The proposed approach is based on representing integers in a residue number system defined by a set of pairwise coprime moduli. Such a decomposition replaces operations on multi-byte integers with independent modular operations on arrays of residues, enabling natural parallelization. Arithmetic operations including addition, subtraction, and multiplication are performed componentwise within the modular system, while reconstruction of the original integer is carried out using algorithms based on the extended Euclidean algorithm. It is shown that in sequential computations the proposed method does not provide a significant advantage over classical arbitrary-precision arithmetic; however, when implemented on GPUs, a substantial performance gain is achieved due to massive parallelism and independence of operations across moduli. An additional advantage is the reduced risk of overflow and more efficient memory access organization enabled by the use of fixed-width integers.

Thus, the use of the CRT constitutes an effective tool for implementing multipoint computations with large integers in supercomputing systems and can be applied to numerical simulation, cryptography, and exhaustive search algorithms requiring high performance and scalability.
\end{abstracteng}

\keywordeng{multi-precision arithmetic}
\keywordeng{supercomputers}
\keywordeng{modular arithmetic}
\keywordeng{parallel computing}

\maketitle

%Раздел обозначается \paragraph, подраздел - \subparagraph (не \section и \subsection)
\paragraph{Введение}


В ряде задач теоретической физики \cite{hao2022cambricon, makarova2023canonical, trukhin2024thermodynamic, trukhin2025low, Vasiliev2020Numerical} данные представляются в виде чисел превышающих стандартные типы языков программирования ($\approx 2^{64}$). Традиционно эту проблема решается представлением чисел в виде строки произвольной длины. Однако такой подход замедляет расчёты, и не работает с графическими процессорами из-за их архитектуры.

В статье представлен алгоритм разложения больших чисел на остатки от деления на набор простых чисел, математические операции над полученными массивами и способ обратной сборки.

\paragraph{Алгоритм работы с длинной арифметикой} 

Принцип работы алгоритма можно разбить на три этапа: 

\begin{enumerate}
	\item Разбиение большого числа на составные части (набор остатков)
	\item Математические операции с массивами остатков от деления на простые числа
	\item Расчёт результата из массива остатков от деления на простые числа
\end{enumerate}

Основой алгоритма является Китайская теорема об остатках (КТО) \cite{Okulov2011, Cormen2001}:

Пусть $a = a_1 a_2 ... a_k$, где $a_i$ попарно взаимно простые натуральные числа, $r = r_1 r_2 ... r_k$ -- натуральные числа удовлетворяющие неравенству $0 \leq r_i \le a_i$. Тогда найдётся такое число
$N$, для которого для всех $i$ выполняется:

\begin{gather}
	\label{eq:rem_th}
	r_i=N \mod a_i
\end{gather}

Из теоремы следует что, для того чтобы разложить число X на остатки $r_i$ необходимо найти остаток от деления на некоторый набор простых чисел $a_i$. Этот набор простых чисел определяет максимальное число $M$, которое мы сможем хранить с использованием КТО. Число $M$ задается следующей формулой:

\begin{gather}
	\label{eq:corollary}
	M = \prod\limits_{i=1}^k a_i, ~~~ M \geqslant X
\end{gather}

Существует несколько подходов к выбору простых чисел для разбиения. 

Первый подход состоит в выборе минимального количества простых чисел, произведение которых превосходит $M$, и который помещаются в максимальный целочисленный тип данных, предоставляемый системой ($2^{32}$ или $2^{64}$). 

Второй подход предлагает использовать множество простых чисел, которые бы помещались в наиболее производительный тип данных. Для графических процессоров (GPU) таким является \textbf{int8}. Несмотря на то, что для второго подхода необходимо производить больше операций за счёт большего количества чисел $a$, применение параллельных вычислений на GPU этот подход даст больший прирост производительности, из-за особенности графических мультипроцессоров.

\paragraph{Математические операции с массивами остатков}

Модульная арифметика \cite{omondi2007residue, soderstrand1986residue} определяет операции сложения, вычитания и умножения следующим образом: если два числа $X_1$ и $X_2$ представимы в виде систем остатков $(r_{11}, r_{12}, ..., r_{1n}) $ и $(r_{21}, r_{22}, ..., r_{2n})$ на модули $(a_1, a_2, ..., a_n)$

\begin{align} 
	X_1 + X_2  = ((r_{11} + r_{21})\ \mathrm{mod}\ a_1, (r_{12} + r_{22})\ \mathrm{mod}\ a_2, ..., (r_{1n} + r_{2n})\ \mathrm{mod}\ a_n), \\
	X_1 - X_2 = ((r_{11} - r_{21})\ \mathrm{mod}\ a_1, (r_{12} - r_{22})\ \mathrm{mod}\ a_2, ..., (r_{1n} - r_{2n})\ \mathrm{mod}\ a_n), \\
	X_1 \cdot X_2 = ((r_{11} \cdot r_{21})\ \mathrm{mod}\ a_1, (r_{12} \cdot r_{22})\ \mathrm{mod}\ a_2, ..., (r_{1n} \cdot r_{2n})\ \mathrm{mod}\ a_n), 
\end{align}
т.е., чтобы сложить, вычесть или умножить 2 числа, достаточно сложить, вычесть или умножить соответствующие элементы векторов остатков этих двух чисел.

В следствии того что: $r_i \in (N \cup 0) $, операция деления не определена для модульной арифметики в явном виде.

\paragraph{Восстановление числа из массива остатков}

Для расшифровки нужно найти модульно обратное $M_i^{-1}\equiv\frac{1}{M_i}\mod a_i$. Для поиска $M_i^{-1}$ используется расширенный алгоритм Евклида.
Расширенный алгоритм Евклида \cite{Okulov2011} является продолжением общего алгоритма Евклида. 

В общем алгоритме для чисел $a$ и $b$ рассчитываются частные $Q_{i-1}$ и остатки $R_i$ такие что $R_0 = a$, $R_1 = b$, $Q_{i-1} = \frac{R_{i-2}}{R_{i-1}}$ и $R_i = R_{i-2} - Q_{i-1} \cdot R_{i-1}$ до тех пор пока $R_i$ не станет равна нулю.

В расширенном алгоритме Евклида добавляются коэффициенты Безу $^1B_{i+1} = ^1B_{i-1}$ $- ^1B_{i}$ и $^2B_{i+1} = ^2B_{i-1} - ^2B_{i}$. Первый коэффициент является модульно обратным для $M_i$ и $a$. Для поиска первого коэффициента был использован алгоритм  \ref{algo:Evclid}.

\begin{algorithm}[H]
	\textbf{Ввод:} Два натуральных числа $a$ и $b$.\\
	\textbf{Вывод:} Первый коэффициент Безу $^1B$.
	\begin{algorithmic}
		\STATE {$R_0 = a$}
		\STATE {$R_1 = b$}
		\WHILE {$R_i > 0$}
			\STATE {$Q_{i-1} = \frac{R_{i-2}}{R_{i-1}}$}
			\STATE {$R_i = R_{i-2} - Q_{i-1} \cdot R_{i-1}$}
			\STATE {$^1B_{i+1} = ^1B_{i-1} - ^1B_{i}$}
		\ENDWHILE {}
		\STATE {$^1B = ^1B_{i+1}$}
	\end{algorithmic}
	\caption{Расширенный алгоритм Евклида}
	\label{algo:Evclid}
\end{algorithm}

Таким образом, для того чтобы расшифровать число обратно используется алгоритм \ref{algo:crt_decription}:

\begin{algorithm}[H]
	\textbf{Ввод:} Массивы простых чисел $a[~]$, и остатков $r[~]$ длины $k$.\\
	\textbf{Вывод:} Расшифрованное число.
	\begin{algorithmic}
		\STATE {Рассчитать $M = \prod\limits_{i=1}^k a_i$}
		\STATE {Рассчитать $M_i = \frac{M}{a_i}$}
		\STATE {Используя расширенный алгоритм Евклида \cite{Okulov2011} найти первый коэффициент Безу $^1B$ для наибольшего общего делителя от $M_i$ и $a_i$:}
		\STATE {Рассчитать искомое число $x = (\sum\limits_{i=1}^{k} r_i \cdot M_i \cdot ^1B)\mod M$}
	\end{algorithmic}
	\caption{Восстановление числа из простых чисел и остатков от делений на них.}
	\label{algo:crt_decription}
\end{algorithm}

В алгоритме \ref{algo:crt_decription} для восстановления числа необходимо вычислить произведение набора остатков на соответствующие $M_i$ и первый коэффициент Безу. Сумма данного произведения, взятая по модулю $M$, и будет равна исходному числу.

\paragraph{Производительность}

На рисунке \ref{fig:timeTest} показана производительность операции сложения массивов чисел фиксированной разрядности порядка $2^{100}$. По оси абсцисс отложена длина массива (число обрабатываемых элементов), по оси ординат — производительность, выраженная в числе операций сложения в секунду, представленная в логарифмическом масштабе.

\begin{figure}[h]\centering
    \centering
    \includegraphics[width=0.48\textwidth]{images/time_sequential18.pdf}
    \hfill
    \includegraphics[width=0.48\textwidth]{images/time_parallel18.pdf}
	\caption{Производительность операции сложения на одном потоке (слева) и при параллельном вычислении (справа).}
	\label{fig:timeTest}
\end{figure}

Левый график соответствует последовательному (однопоточному) режиму выполнения. Видно, что в этом случае предложенный метод на основе китайской теоремы об остатках не демонстрирует преимущества по сравнению с классической длинной арифметикой, реализованной в библиотеке GMP. Это объясняется тем, что при отсутствии параллелизма затраты на запуск ядра GPU компенсируют преимущество использования более простых модульных операций.

Правый график иллюстрирует результаты при параллельном выполнении вычислений. В этом режиме метод КТО демонстрирует кратное преимущество, которое возрастает с увеличением длины массива. Для больших объёмов данных производительность предложенного подхода превышает показатели реализации на базе GMP более чем в 100 раз. Такой рост производительности обусловлен независимостью вычислений как по элементам массива, так и по модулям, что позволяет эффективно задействовать вычислительные ресурсы GPU.

\paragraph{Заключение}

Предлагаемый подход обладает рядом преимуществ, которые делают его эффективным для использования в вычислениях на GPU. 

Во-первых, независимые параллельные вычисления с использованием алгоритма КТО достигаются за счёт работы с отдельными остатками. Это существенно ускоряет обработку массивов остатков на разных потоках или ядрах GPU в переборном алгоритме декомпозиции \cite{trukhin2025alg}.

Вторым важным аспектом является решение проблемы переполнения.

Кроме того, использование КТО способствует оптимизации доступа к памяти GPU. Применение модульной арифметики позволяет размещать данные в оптимальном для архитектуры GPU формате, что улучшает производительность ввода-вывода и снижает задержки доступа к памяти.

Несмотря на преимущества, подход с использованием КТО имеет ряд недостатков. Восстановление исходного числа после операций требует значительных ресурсов и может замедлить работу из-за синхронизации на GPU. Алгоритм с использованием КТО является избыточным при работе с числами, сопоставимым с простыми числами, используемыми для кодировки.

Таким образом, алгоритм с использованием КТО является мощным инструментом для работы с многоразрядными числами на GPU. Его применение позволяет эффективно реализовывать параллелизм, избежать переполнения и оптимизировать использование памяти. В задачах, требующих высокопроизводительных вычислений, таких как алгоритмы полного перебора и численные расчеты, КТО может существенно ускорить процесс обработки данных и повысить производительность системы.

%%%%пример оформления списка литературы в соответствии с ГОСТ Р 7.0.5-2008.
	
\begin{thebibliography}{99}

\bibitem[Васильев и др., 2020]{Vasiliev2020Numerical} {\it Васильев~Е.\,В., Пержу~А.\,В., Король~А.\,О., Капитан~Д.\,Ю., Рыбин~А.\,Е., Солдатов~К.\,С., Капитан~В.\,Ю.} Численное моделирование двумерных магнитных скирмионных структур // Компьютерные исследования и моделирование.~--- 2020.~--- Т.~12, \No~5.~--- С.~1051--1061.
\\
{\footnotesize{\it Vasil'ev~E.\,V., Perzhu~A.\,V., Korol'~A.\,O., Kapitan~D.\,Yu., Rybin~A.\,E., Soldatov~K.\,S., Kapitan~V.\,Yu.} Chislennoe modelirovanie dvumernykh magnitnykh skirmionnykh struktur [Numerical simulation of two-dimensional magnetic skyrmion structures] // Komp'yuternye issledovaniya i modelirovanie.~--- 2020.~--- Vol.~12, No.~5.~--- S.~1051--1061 (in Russian).\par}

\bibitem[Окулов, Лялин, 2011]{Okulov2011} {\it Окулов~С.\,М., Лялин~А.\,В.} Расширенный алгоритм Евклида // Информатика и образование.~--- 2011.~--- \No~8.~--- С.~37--41.
\\
{\footnotesize{\it Okulov~S.\,M., Lyalin~A.\,V.} Rasshirennyi algoritm Evklida [Extended Euclidean algorithm] // Informatika i obrazovanie.~--- 2011.~--- No.~8.~--- S.~37--41 (in Russian).\par}

\bibitem[Трухин и др., 2025]{trukhin2025alg} {\it Трухин~В.\,О., Лобанова~Э.\,А., Анисич~А.\,И., Макарова~К.\,В., Макаров~А.\,Г., Нефедев~К.\,В.} Переборный алгоритм декомпозиции для решения модели Изинга // Дальневосточный математический журнал.~--- 2025.~--- Т.~25, \No~1.~--- C.~102--112.
\\
{\footnotesize{\it Trukhin~V.\,O., Lobanova~E.\,A., Anisich~A.\,I., Makarova~K.\,V., Makarov~A.\,G., Nefedev~K.\,V.} Perebornyi algoritm dekompozitsii dlya resheniya modeli Izinga [An enumeration decomposition algorithm for solving the Ising model] // Dal'nevostochnyi matematicheskii zhurnal.~--- 2025.~--- Vol.~25, No.~1.~--- S.~102--112 (in Russian).\par}

\bibitem[Cormen et al., 2022]{Cormen2022} {\it Cormen~T.\,H., Leiserson~C.\,E., Rivest~R.\,L., Stein~C.} Introduction to algorithms.~--- MIT Press, 2022.\\

\bibitem[Hao et al., 2022]{hao2022cambricon}
{\it Hao~Y., Zhao~Y., Liu~C., Du~Z., Cheng~S., Li~X., Hu~X., Guo~Q., Xu~Z., Chen~T.} Cambricon-p: A bitflow architecture for arbitrary precision computing // 2022 55th IEEE/ACM International Symposium on Microarchitecture (MICRO).~--- IEEE, 2022.~--- P.~57--72.\\

\bibitem[Makarova et al., 2023]{makarova2023canonical} {\it Makarova~K., Makarov~A., Strongin~V., Titovets~I., Shevchenko~Y., Kapitan~V., Rybin~A., Kapitan~D., Korol~A., Vasiliev~E., Ovchinnikov~P., Soldatov~K., Trukhin~V., Nefedev~K.} Canonical Monte Carlo multispin cluster method // Journal of Computational and Applied Mathematics.~--- 2023.~--- Vol.~427.~--- P.~115153.\\

\bibitem[Omondi, Premkumar, 2007]{omondi2007residue} {\it Omondi~A.\,R., Premkumar~A.\,B.} Residue number systems: theory and implementation.~--- World Scientific, 2007.~--- Vol.~2.\\

\bibitem[Soderstrand et al., 1986]{soderstrand1986residue} {\it Soderstrand~M.\,A., Jenkins~W.\,K., Jullien~G.\,A., Taylor~F.\,J.} Residue number system arithmetic: modern applications in digital signal processing.~--- IEEE Press, 1986.\\

\bibitem[Trukhin et al., 2024]{trukhin2024thermodynamic} {\it Trukhin~V., Strongin~V., Chesnokov~M., Makarov~A., Lobanova~E., Shevchenko~Y., Nefedev~K.} Thermodynamic equilibrium of $\pm J$ Ising model on square lattice~// Physica A: Statistical Mechanics and its
Applications.~--- 2024.~--- Vol.~655.~--- P.~130172.\\

\bibitem[Trukhin et al., 2025]{trukhin2025low} {\it Trukhin~V., Prokhorov~E., Makarov~A., Nefedev~K.} Low-temperature states of the Ising $\pm J$ model on a square lattice~// Physica A: Statistical Mechanics and its Applications.~--- 2025.~--- P.~130729.\\

\end{thebibliography}

\end{document}
