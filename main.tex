\documentclass{crm-article}

\begin{document}

\journalVol{10}
\journalNo{1} %выпуска
\setcounter{page}{1}

% раздел журнала
\journalSection{Математические основы и численные методы моделирования}
\journalSectionEn{Mathematical modeling and numerical simulation}


% дата получения
\journalReceived{01.06.2016.}
%\journalReviewed{01.06.2016.}
%принято к публикации
\journalAccepted{01.06.2016.}


\UDC{519.8}
\title{Применение китайской теоремы об остатках для работы с длинной арифметикой}
\titleeng{Title}
\thanks{Исследование выполнено за счет гранта Российского научного фонда № 24-71-10069, \linebreak https://rscf.ru/project/24-71-10069/ \linebreak}
\thankseng{The research was supported by the Russian Science Foundation grant No. 24-71-10069, \linebreak https://rscf.ru/en/project/24-71-10069/ \linebreak}

\author[1,2]{\firstname{В.\,О.}~\surname{Трухин}}
\authorfull{Вячеслав Олегович Трухин}
\authoreng{\firstname{V.\,O.}~\surname{Trukhin}}
\authorfulleng{Viacheslav O. Trukhin}
\email{slavatruhin@gmail.com}
\affiliation[1]{Департамент теоретической физики и интеллектуальных технологий, Институт наукоемких технологий и передовых материалов, Дальневосточный федеральный университет,\protect\\ 690922, Россия, г. Владивосток, о. Русский, п. Аякс, 10}
\affiliation[2]{Институт прикладной математики, Дальневосточное отделение Российской академии наук,\protect\\ 690041, Россия, г. Владивосток, Ул. Радио д. 7}
\affiliationeng[1]{Department of theoretical physics and intellectual technologys, Institute of scientific and advanced materials, Far eastern federal university,\protect\\ Address}
\affiliationeng[2]{Institute for Applied Mathematics, Far Eastern Branch, Russian Academy of Sciences,\protect\\ Address}

\author[1,2]{\firstname{В.\,С.}~\surname{Стронгин}}
\authorfull{Владислав Сергеевич Стронгин}
\authoreng{\firstname{V.\,O.}~\surname{Strongin}}
\authorfulleng{Vladislav S. Strongin}

\author[1,2]{\firstname{Э.\,А.}~\surname{Лобанова}}
\authorfull{Элиза Александровна Лобанова}
\authoreng{\firstname{E.\,A.}~\surname{Lobanova}}
\authorfulleng{Eliza A. Lobanova}

\author[1,2]{\firstname{А.\,Г.}~\surname{Макаров}}
\authorfull{Александр Геннадиевич Макаров}
\authoreng{\firstname{V.\,O.}~\surname{Makarov}}
\authorfulleng{Alexander G. Makarov}

\author[1,2]{\firstname{К.\,В.}~\surname{Нефедев}}
\authorfull{Нефедев Константин Валентинович}
\authoreng{\firstname{K.\,V.}~\surname{Nefedev}}
\authorfulleng{Konstantin V. Nefedev}

\begin{abstract}
В работе рассматривается применение Китайской теоремы об остатках (КТО) для оптимизации вычислений с большими числами в суперкомпьютерах, в частности при использовании графических процессоров. Предложенный метод предполагает разложение больших чисел на остатки по отношению к набору простых чисел, что способствует эффективному выполнению арифметических операций, таких как сложение, вычитание и умножение, и одновременно снижает риск переполнения памяти. Преимущество такого подхода заключается в возможности применения параллельных вычислений, что значительно повышает скорость вычислений и повышает скорость чтения/записи памяти.
\end{abstract}


\keyword{большие числа}
\keyword{суперкомпьютеры}
\keyword{модульная арифметика}
\keyword{параллельные вычисления}

\begin{abstracteng}
The present study examines the application of the Chinese Remainder Theorem (CRT) to optimize large-integer computations in supercomputing environments, particularly when employing graphics processing units (GPUs). The proposed method involves decomposing large numbers into residues with respect to a set of prime moduli, thereby enabling efficient execution of arithmetic operations such as addition, subtraction, and multiplication, while simultaneously reducing the risk of memory overflow. A key advantage of this approach lies in its suitability for parallel computation, which substantially increases computational throughput and improves memory read/write performance.
\end{abstracteng}
\keywordeng{multi-precision arithmetic}
\keywordeng{supercomputers}
\keywordeng{modular arithmetic}
\keywordeng{parallel computing}

\maketitle

%Раздел обозначается \paragraph, подраздел - \subparagraph (не \section и \subsection)
\paragraph{Введение}

Текст статьи. Пожалуйста, внимательно прочитайте руководство для авторов на сайте журнала: \url{http://crm-en.ics.org.ru/journal/page/authors/}.


%%%%%%%%%%%%%%%%%%

Доступные окружения:
teo -- Теорема,
hyp -- Гипотеза,
lem -- Лемма,
cor -- Следствие,
pro -- Предложение,
con -- Предположение,
fed -- Определение,
rem -- Замечание,
com -- Комментарий,
exl -- Пример,
sol -- Решение.

%%% выражения типа "то есть", "так далее" не сокращаются, а пишутся полностью.
%%% обратите внимание, что в выключных формулах пределы интегрирования
%%% проставляются сверху, а не справа, то есть используется команда \limits.
%%%
%%% окружение gather (в том числе в варианте со звездочкой) предпочтительнее
%%% окружения equation и "двойных" долларов


\begin{gather} \label{1}  %%%%%%%%%
\dot x(t)=\int\limits_{-r}^0 dA(t,s)x(t+s),\quad  t\in\mathbb R=(-\infty,\infty).
\end{gather}

В дальнейшем систему \eqref{1} будем отождествлять \ldots
\begin{gather} \label{2}  %%%%%%%%%
\dot y(t)=\int\limits_{-r}^0 dB(t,s)y(t+s),\quad  t\in\mathbb R=(-\infty,\infty).
\end{gather}


%Оформление теорем с нумерацией
\begin{teo}
Текст.
\end{teo}

\proof*%Доказательство 1 вариант
Текст доказательства теоремы.\qed

%Оформление теорем без нумерации
%(вариант со звездочкой возможен и для оформления гипотез, лемм, следствий, предложений, предположений,
%определений, замечаний, комментариев, примеров, решений)
\begin{teo*}
Текст.
\end{teo*}

%Оформление гипотез
\begin{hyp}
Текст.
\end{hyp}

%Оформление лемм
\begin{lem}
Текст.
\end{lem}

%Оформление следствий
\begin{cor}
Текст.
\end{cor}

%Оформление предложений
\begin{pro}
Текст.
\end{pro}

%Оформление предположений
\begin{con}
Текст.
\end{con}

%Оформление определений
\begin{fed}
Текст.
\end{fed}

%Оформление замечаний
\begin{rem}
Текст.
\end{rem}

%Оформление следствий
\begin{com}
Текст.
\end{com}

%Оформление примеров
\begin{exl}
Текст.
\end{exl}

%Оформление решений
\begin{sol}
Текст.
\end{sol}


%Оформление рисунков
%готовые рисунки в формате eps необходимо сохранять в папку eps,
%файлы CorelDraw - в папку cdr (если они имеются).
\begin{figure}[!ht]
\centering
%\includegraphics{eps/autor_01.eps}
\smallskip
\caption{Рисунки должны иметь содержательные подписи, позволяющие понять их смысл без обращения к основному тексту статьи. Все используемые обозначения должны быть расшифрованы в подписи, координатные оси на графиках подписаны (с указанием единиц измерения). Все подписи, обозначения, легенды и т.п. должны быть выполнены на том же языке, что и сама статья.}
\end{figure}


%Оформление таблиц
\begin{table}[!ht]
\centering
\caption{Таблицы должны иметь содержательные подписи, позволяющие понять их смысл без обращения к основному тексту статьи.}
\medskip
\begin{tabular}{|c|c|c|}
\hline
Строка 1 & Колонка 2 & Колонка 3\\
\hline
Строка 2& Колонка 2 & Колонка 3\\
\hline
\end{tabular}
\end{table}

%Пример выключенной формулы
$$
Y= S_1\times S_2\times \ldots \times S_N.
$$


%%%%пример оформления списка литературы в соответствии с ГОСТ Р 7.0.5-2008.
	
\begin{thebibliography}{99}


\bibitem[Биррелл, Девис, 1984]{Devis} {\it Биррелл~Н., Девис~В.}
    Квантованные поля в искривленном пространстве-времени.~--- М.:
    Мир, 1984.~--- 356~с.\\
%\bibitem[Birrell, Devis, 1984]{devisENG}
{\footnotesize {\em Birrell N. D., Davies P. C. W.} Quantum fields in curved space.~--- Cambridge university press, 1984.~--- No.~7.
(Russ. ed.: {\it Birrell~N., Devis~V.}
    Kvantovannye polya v iskrivlennom prostranstve vremeni.~--- Moskva:
    Mir, 1984~--- 356~s.) \par}

\bibitem[Браун, 1983]{Braun} {\it Браун~П.\,А., Киселев~А.\,А.}
    Введение в теорию молекулярных спектров.~--- Л.: Изд-во Ленингр.
    ун-та, 1938.~--- 232~с.
\\
%\bibitem[Braun, 1983]{BraunENG}
{\footnotesize{\it Braun~P.\,A., Kiselev~A.\,A.}
    Vvedenie v teoriyu molekulyarnyh spektrov [Introduction to the Theory of Molecular Spectra].~--- Leningrad: Izd-vo Leningr.
    un-ta, 1938.~--- 232~s. (in Russian).\par}

\bibitem[Бреев~А.\,И., 2007]{Br07} {\it Бреев~А.\,И., Широков~И.\,В.,
    Разумов~Н.} Поляризация вакуума скалярного поля на многообразии,
    конформно-эквивалентном RхG // Известия высших учебных заведений.
    Физика.~--- 2007.~--- \No~10.~--- C.~50--56.
\\
%\bibitem[breev~A.\,I.,2007]{Br07ENG}
{\footnotesize{\it Breev~A.\,I.,
    Shirokov~I.\,V., Razumov N.} Polyarizaciya vakuuma skalyarnogo
    polya na mnogoobrazii konformno-ehkvivalentnom rhg [Polarization
    of a scalar field vacuum on a manifold conformally equivalent to
    the manifold R$\otimes$ G] // Izvestiya
    vysshih uchebnyh zavedenij. Fizika.~--- 2007.~--- No.~10.~--- S.~50--56 (in Russian).\par}

\bibitem[Гончаровский, 2009]{Gon09} {\it Гончаровский~М.\,М.,
    Широков~И.\,В.} Интегрируемый класс дифференциальных уравнений с
    нелокальной нелинейностью на группах Ли~// Теоретическая и
    математическая физика.~--- Т.~161, \No~3.~--- C. 332--345.
\\
%\bibitem[Goncharovskij,2009]{Gon09ENG}
{\footnotesize{\it Goncharovskij~M.\,M., Shirokov~I.\,V.} Integriruemyj klass differencialnyh uravnenij s
    nelokalnoj nelinejnostyu nagruppah Li [An integrable class of
    differential equations with nonlocal nonlinearity on Lie groups]~// Teoreticheskaya i
    Matematicheskaya Fizika.~--- Vol.~161, No.~3.~--- S.~332--345 (in Russian).\par}



\bibitem[Гриб, Мамаев, 1988]{Grib} {\it Гриб~A.\,A., Мамаев~C.\,Г.,
    Мостепаненко~В.\,М.} Вакуумные квантовые эффекты в сильных
    полях.~--- М.: Атомиздат, 1988.~--- 288~с.
\\
%\bibitem[Grib, Mamaev, 1988]{GribENG}
{\footnotesize{\it Grib~A.\,A.,
    Mamaev~C.\,G., Mostepanenko~V.\,M.} Vakuumnye kvantovye ehffekty
    v silnyh polyah [Vacuum quantum effects in strong fields].~--- Moskva: Atomizdat, 1988.~--- 288~s. (in Russian).\par}

\bibitem[Кириллов, 1978]{kirr} {\it Кириллов~А.\,А.} Элементы теории
    представлений.~--- М.: Наука, 1978.~--- 344~с.
\\
%\bibitem[Kirillov, 1978]{kirrENG}
{\footnotesize{\it Kirillov~A.\,A.} Ehlementy
    teorii predstavlenij [Elements of the Theory of Representations].~--- Moskva: Nauka, 1978.~--- 344~s. (in Russian).\par}

\bibitem[Шаповалов, 1995]{Shap95} {\it Шаповалов~А.\,В.,
    Широков~И.\,В.} Некоммутативное интегрирование линейных
    дифференциальных уравнений  // Теоретическая и математическая
    физика.~--- 1995.~--- Т.~104, \No~2.~--- С.~195--213.
\\
%\bibitem[Shapovalov, 9915]{Shap95ENG}
{\footnotesize{\it Shapovalov~A.\,V.,
    Shirolov~I.\,V.} Nekommutativnoe integrirovanie linejnyh
    differencialnyh uravnenij [Noncommutative integration of linear
    differential equations]~// Teoreticheskaya i matematicheskaya
    fizika.~--- 1995.~--- Vol.~104, No.~2.~--- S.~195--213 (in Russian).\par}


\bibitem[Breev, 2014]{Br14} {\it Breev~A.\,I. Shapovalov~A.\,V.}
    Yang-Mills gauge felds conserving the symmetry algebra of the
    Dirac equation in a homogeneous space~// Journal of Physics:
    Conference Series.~--- 2014.~--- Vol.~563.~--- P.~012004.

\bibitem[Hu, 1973]{Hu1} {\it Hu~B.\,L.} Scalar waves in the Mixmaster
    Universe. I. The Helmholtz equation in a fixed background //
    Phys. Rev.~D.~--- 1973.~--- Vol.~{8}, No.~4.~--- P.~1048--1060.

\bibitem[Hu, 1974]{Hu2} {\it Hu~B.\,L.} Scalar waves in the Mixmaster
    Universe. II. Particle creation~// Phys. Rev.~D.~--- 1974.~--- Vol.~9,
No.~9.~--- P.~3263--3281.

\bibitem[Pritomanov, 1985]{Prit} {\it Pritomanov~S.\,A.} Quantum
    effects in Mixmaster Universe // Phys. Lett.~A.~--- 1985.~--- Vol.~107,
No.~1.~--- P.~33--35.

\bibitem[Ryan, 1975]{Ryan} {\it Ryan~M.\,P., Shepley~L.\,C.}
    Homogeneous relativistic cosmologies.~--- Princeton: Princeton
    series in Physics, 1975.~--- 336~p.


\end{thebibliography}


\end{document}
