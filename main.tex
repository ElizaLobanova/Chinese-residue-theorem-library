\documentclass[10pt]{article}
\usepackage{femj_ru}
\usepackage{algorithm}
\usepackage{algorithmic}

\begin{document}
\newcommand{\todo}[1]{\textbf{#1}}


\Pages(0--0)

\def\Im{\mathop{\mathrm{Im}}\nolimits}

\summary Trukhin~V.\,O.$^{1, 2}$\!, Strongin~V.\,S.$^{1, 2}$\!, Lobanova~E.\,A.$^{1, 2}$\, Nefedev~K.\,V.$^{1, 2}$\author

Application of the Chinese residue theorem for dealing with multi-precision numbers in supercomputing\title

The study explores the application of the Chinese Remainder Theorem (CRT) to optimize large-number computations in supercomputing, particularly when employing graphics processing units (GPUs). The proposed method decomposes large numbers into residues with respect to a set of prime moduli, enabling efficient execution of arithmetic operations such as addition, subtraction, and multiplication while simultaneously reducing the risk of memory overflow. The main advantage of this approach lies in its suitability for parallel computation, which substantially increases computational speed and improves memory read/write performance. However, reconstructing the original number from its residues requires considerable computational resources, which may potentially diminish overall performance.

\keywords{multi-precision arithmetic, supercomputers, modular arithmetic, parallel computing.}
\org{$^1$ Institute for Applied Mathematics, Far Eastern Branch, Russian Academy of Sciences\\ 
$^2$ Department of theoretical physics, Far eastern federal university, Russia}
\references{ %транслитерированный список литературы

}

\UDC{519.17+519.11+519.684+519.683}
\AMS{68R10 + 82B30 + 65Y05}

\SupportedBy{Исследование выполнено за счет гранта Российского научного фонда № 24-71-10069, \linebreak https://rscf.ru/project/24-71-10069/\linebreak}

\submitted{ октября 2024 г.}

\title{Применение китайской теоремы об остатках для работы с длинной арифметикой}

\author[1,2]{В.\, О.~Трухин}{Департамент теоретической физики и интеллектуальных технологий, Институт наукоемких технологий и передовых материалов, Дальневосточный федеральный университет. 690922, Россия, г. Владивосток, о. Русский, п. Аякс, 10}{trukhin.vo@dvfu.ru}
\author[1,2]{В.\, С.~Стронгин}{Институт прикладной математики, Дальневосточное отделение Российской академии наук. 690041, Россия, г. Владивосток, Ул. Радио д. 7}{}
\author[1,2]{Э.\, А.~Лобанова}{}{}
\author[1,2]{К.\, В.~Нефедев}{}{}




\makeface

%\markright {Разбавленная модель кубического спинового льда ...} %Добавляем, если длинное название статьи
\markleft{В.\, О.~Трухин, В.\, С. Стронгин, Э.\,А.~Лобанова, ...} %добавляем, если много авторов


\abstract В работе рассматривается применение Китайской теоремы об остатках (КТО) для оптимизации вычислений с большими числами в суперкомпьютерах, в частности при использовании графических процессоров. Предложенный метод предполагает разложение больших чисел на остатки по отношению к набору простых чисел, что способствует эффективному выполнению арифметических операций, таких как сложение, вычитание и умножение, и одновременно снижает риск переполнения памяти. Преимущество такого подхода заключается в возможности применения параллельных вычислений, что значительно повышает скорость вычислений и повышает скорость чтения/записи памяти.

\keywords{большие числа, суперкомпьютеры, модульная арифметика, параллельные вычисления.}

\DOI{to be presented}

\section*{Введение}

В ряде задач теоретической физики \cite{hao2022cambricon, makarova2023canonical, trukhin2024thermodynamic, trukhin2025low} данные представляются в виде чисел превышающих стандартные типы языков программирования ($\approx 2^{64}$). Традиционно эту проблема решается представлением чисел в виде строки произвольной длины. Однако такой подход замедляет расчёты, и не работает с графическими процессорами из-за их архитектуры.

В статье представлен алгоритм разложения больших чисел на остатки от деления на набор простых чисел, математические операции над полученными массивами и способ обратной сборки.

\section{Алгоритм работы с длинной арифметикой} 

Принцип работы алгоритма можно разбить на три этапа: 

\begin{enumerate}
	\item Разбиение большого числа на составные части (набор остатков)
	\item Математические операции с массивами остатков от деления на простые числа
	\item Расчёт результата из массива остатков от деления на простые числа
\end{enumerate}

Основой алгоритма является Китайская теорема об остатках (КТО) \cite{Okulov2011,Cormen2001}:

Пусть $a = a_1 a_2 ... a_k$, где $a_i$ попарно взаимно простые натуральные числа, $r = r_1 r_2 ... r_k$ -- натуральные числа удовлетворяющие неравенству $0 \leq r_i \le a_i$. Тогда найдётся такое число
$N$, для которого для всех $i$ выполняется:

\begin{equation}
	r_i=N \mod a_i
	\label{eq:rem_th}
\end{equation}

Из теоремы следует что, для того чтобы разложить число X на остатки $r_i$ необходимо найти остаток от деления на некоторый набор простых чисел $a_i$. Этот набор простых чисел определяет максимальное число $M$, которое мы сможем хранить с использованием КТО. Число $M$ задается следующей формулой:

\begin{equation}
	M = \prod\limits_{i=1}^k a_i, ~~~ M \geqslant X
	\label{eq:corollary}
\end{equation}

Существует несколько подходов к выбору простых чисел для разбиения. 

Первый подход состоит в выборе минимального количества простых чисел, произведение которых превосходит $M$, и который помещаются в максимальный целочисленный тип данных, предоставляемый системой ($2^{32}$ или $2^{64}$). 

Второй подход предлагает использовать множество простых чисел, которые бы помещались в наиболее производительный тип данных. Для графических процессоров (GPU) таким является \textbf{int8}. Несмотря на то, что для второго подхода необходимо производить больше операций за счёт большего количества чисел $a$, применение параллельных вычислений на GPU этот подход даст больший прирост производительности, из-за особенности графических мультипроцессоров.

\section{Математические операции с массивами остатков}

Модульная арифметика \cite{omondi2007residue, soderstrand1986residue} определяет операции сложения, вычитания и умножения следующим образом: если два числа $X_1$ и $X_2$ представимы в виде систем остатков $(r_{11}, r_{12}, ..., r_{1n}) $ и $(r_{21}, r_{22}, ..., r_{2n})$ на модули $(a_1, a_2, ..., a_n)$

\begin{align} 
	X_3 &= X_1 + X_2 = \nonumber \\ 
	& = ((r_{11} + r_{21})\ \mathrm{mod}\ a_1, (r_{12} + r_{22})\ \mathrm{mod}\ a_2, ..., (r_{1n} + r_{2n})\ \mathrm{mod}\ a_n), \\
	X_4 &= X_1 - X_2 = \nonumber \\ 
	& =((r_{11} - r_{21})\ \mathrm{mod}\ a_1, (r_{12} - r_{22})\ \mathrm{mod}\ a_2, ..., (r_{1n} - r_{2n})\ \mathrm{mod}\ a_n), \\
	X_5 &= X_1 \cdot X_2 = \nonumber \\ 
	& = ((r_{11} \cdot r_{21})\ \mathrm{mod}\ a_1, (r_{12} \cdot r_{22})\ \mathrm{mod}\ a_2, ..., (r_{1n} \cdot r_{2n})\ \mathrm{mod}\ a_n), 
\end{align}
т.е., чтобы сложить, вычесть или умножить 2 числа, достаточно сложить, вычесть или умножить соответствующие элементы векторов остатков этих двух чисел.

В следствии того что: $r_i \in (N \cup 0) $, операция деления не определена для модульной арифметики в явном виде.

\section{Восстановление числа из массива остатков}

Для расшифровки нужно найти модульно обратное $M_i^{-1}\equiv\frac{1}{M_i}\mod a_i$. Для поиска $M_i^{-1}$ используется расширенный алгоритм Евклида.
Расширенный алгоритм Евклида \cite{Okulov2011} является продолжением общего алгоритма Евклида. 

В общем алгоритме для чисел $a$ и $b$ рассчитываются частные $Q_{i-1}$ и остатки $R_i$ такие что $R_0 = a$, $R_1 = b$, $Q_{i-1} = \frac{R_{i-2}}{R_{i-1}}$ и $R_i = R_{i-2} - Q_{i-1} \cdot R_{i-1}$ до тех пор пока $R_i$ не станет равна нулю.

В расширенном алгоритме Евклида добавляются коэффициенты Безу $^1B_{i+1} = ^1B_{i-1} - ^1B_{i}$ и $^2B_{i+1} = ^2B_{i-1} - ^2B_{i}$. Первый коэффициент является модульно обратным для $M_i$ и $a$. Для поиска первого коэффициента был использован алгоритм (алг. \ref{algo:Evclid}).


\begin{algorithm}[H]
	\textbf{Ввод:} Два натуральных числа $a$ и $b$.\\
	\textbf{Вывод:} Первый коэффициент Безу $^1B$.
	\begin{algorithmic}
		\STATE {$R_0 = a$}
		\STATE {$R_1 = b$}
		\WHILE {$R_i > 0$}
			\STATE {$Q_{i-1} = \frac{R_{i-2}}{R_{i-1}}$}
			\STATE {$R_i = R_{i-2} - Q_{i-1} \cdot R_{i-1}$}
			\STATE {$^1B_{i+1} = ^1B_{i-1} - ^1B_{i}$}
		\ENDWHILE {}
		\STATE {$^1B = ^1B_{i+1}$}
	\end{algorithmic}
	\caption{Расширенный алгоритм Евклида}
	\label{algo:Evclid}
\end{algorithm}


Таким образом, для того чтобы расшифровать число обратно используется следующий алгоритм:

\begin{algorithm}[H]
	\textbf{Ввод:} Массивы простых чисел $a[~]$, и остатков $r[~]$ длины $k$.\\
	\textbf{Вывод:} Расшифрованное число.
	\begin{algorithmic}
		\STATE {Рассчитать $M = \prod\limits_{i=1}^k a_i$}
		\STATE {Рассчитать $M_i = \frac{M}{a_i}$}
		\STATE {Используя расширенный алгоритм Евклида \cite{Okulov2011} найти первый коэффициент Безу $^1B$ для наибольшего общего делителя от $M_i$ и $a_i$:}
		\STATE {Рассчитать искомое число $x = (\sum\limits_{i=1}^{k} r_i \cdot M_i \cdot ^1B)\mod M$}
	\end{algorithmic}
	\caption{Восстановление числа из простых чисел и остатков от делений на них.}
	\label{algo:crt_decription}
\end{algorithm}

В данном алгоритме для восстановления числа необходимо вычислить произведение набора остатков на соответствующие $M_i$ и первый коэффициент Безу. Сумма данного произведения, взятая по модулю $M$, и будет равна исходному числу.

\section{Заключение}

Предлагаемый подход обладает рядом преимуществ, которые делают его эффективным для использования в вычислениях на GPU. 

Во-первых, независимые параллельные вычисления с использованием алгоритма КТО достигается за счёт работы с отдельными остатками. Это существенно ускоряет обработку массивов остатков на разных потоках или ядрах GPU в переборном алгоритме декомпозиции \cite{trukhin2025alg}.

Вторым важным аспектом является решение проблемы переполнения.

Кроме того, использование КТО способствует оптимизации доступа к памяти GPU. Применение модульной арифметики позволяет размещать данные в оптимальном для архитектуры GPU формате, что улучшает производительность ввода-вывода и снижает задержки доступа к памяти.

Несмотря на преимущества, подход с использованием КТО имеет ряд недостатков. Восстановление исходного числа после операций требует значительных ресурсов и может замедлить работу из-за синхронизации на GPU. Алгоритм с использованием КТО является избыточным при работе с числами, сопоставимым с простыми числами, используемыми для кодировки.

Таким образом, алгоритм с использованием КТО является мощным инструментом для работы с большими числами на GPU. Его применение позволяет эффективно реализовывать параллелизм, избежать переполнения и оптимизировать использование памяти. В задачах, требующих высокопроизводительных вычислений, таких как алгоритмы полного перебора и численные расчеты, КТО может существенно ускорить процесс обработки данных и повысить производительность системы.
%\vspace{15mm}

\begin{thebibliography}{20}
\setlength{\parsep}{0pt}\setlength{\itemsep}{3pt}

\bibitem{hao2022cambricon}
\by Hao, Yifan and Zhao, Yongwei and Liu, Chenxiao and Du, Zidong and Cheng, Shuyao and Li, Xiaqing and Hu, Xing and Guo, Qi and Xu, Zhiwei and Chen, Tianshi
\jour 2022 55th IEEE/ACM International Symposium on Microarchitecture (MICRO)
\paper Cambricon-P: A bitflow architecture for arbitrary precision computing
\pages 57-72
\yr 2022

\bibitem{makarova2023canonical}
\by Makarova, Kseniia and Makarov, Aleksandr and Strongin, Vladislav and Titovets, Iuliia and Shevchenko, Yuriy and Kapitan, Vitalii and Rybin, Alexey and Kapitan, Dmitrii and Korol, Alena and Vasiliev, Egor and others
\jour Journal of Computational and Applied Mathematics
\paper Canonical Monte Carlo multispin cluster method
\pages 115153
\yr 2023

\bibitem{trukhin2024thermodynamic}
\by Trukhin, Viacheslav Olegovich and Strongin, Vladislav Sergeevich and Chesnokov, Mikhail Alexeyevich and Makarov, Aleksandr Gennadievich and Lobanova, Eliza Alexandrovna and Shevchenko, Yuriy Andreevich and Nefedev, Konstantin Valentinovich
\jour Physica A: Statistical Mechanics and its Applications
\paper Thermodynamic equilibrium of$\pm$J Ising model on square lattice
\pages 130172
\yr 2024

\bibitem{trukhin2025low}
\by Trukhin, VO and Prokhorov, EI and Makarov, AG and Nefedev, KV
\jour Physica A: Statistical Mechanics and its Applications
\paper Low-temperature states of the Ising$\pm$J model on a square lattice
\pages 130729
\yr 2025

\bibitem{Okulov2011}
\by С. М. Окулов, А. В. Лялин
\jour Информатика и образование
\paper Расширенный алгоритм Евклида
\issue 5
\pages 37-41
\yr 2011

\bibitem{Cormen2001}
\by H. Cormen, E. Leiserson, L. Rivest, C. Stein
\jour The MIT Press 
\paper Introduction to Algorithms, Second Edition
\pages 950-954
\yr 2001


\bibitem{omondi2007residue}
\by Omondi, Amos R and Premkumar, A Benjamin
\jour Physics Letters A
\paper Residue number systems: theory and implementation
\vol 2
\yr 2007


\bibitem{soderstrand1986residue}
\by Soderstrand, Michael A and Jenkins, W Kenneth and Jullien, Graham A and Taylor, Fred J
\jour Physics Letters A
\paper Residue number system arithmetic: modern applications in digital signal processing
\yr 1986

\bibitem{trukhin2025alg}
\by В. О. Трухин и др.
\jour Дальневосточный математический журнал
\paper Переборный алгоритм декомпозиции для решения модели Изинга
\vol 25
\yr 2025


\end{thebibliography}

\EndArticle

\end{document} 
